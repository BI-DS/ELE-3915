% Options for packages loaded elsewhere
\PassOptionsToPackage{unicode}{hyperref}
\PassOptionsToPackage{hyphens}{url}
\PassOptionsToPackage{dvipsnames,svgnames,x11names}{xcolor}
%
\documentclass[
  letterpaper,
  DIV=11,
  numbers=noendperiod]{scrreprt}

\usepackage{amsmath,amssymb}
\usepackage{lmodern}
\usepackage{iftex}
\ifPDFTeX
  \usepackage[T1]{fontenc}
  \usepackage[utf8]{inputenc}
  \usepackage{textcomp} % provide euro and other symbols
\else % if luatex or xetex
  \usepackage{unicode-math}
  \defaultfontfeatures{Scale=MatchLowercase}
  \defaultfontfeatures[\rmfamily]{Ligatures=TeX,Scale=1}
\fi
% Use upquote if available, for straight quotes in verbatim environments
\IfFileExists{upquote.sty}{\usepackage{upquote}}{}
\IfFileExists{microtype.sty}{% use microtype if available
  \usepackage[]{microtype}
  \UseMicrotypeSet[protrusion]{basicmath} % disable protrusion for tt fonts
}{}
\makeatletter
\@ifundefined{KOMAClassName}{% if non-KOMA class
  \IfFileExists{parskip.sty}{%
    \usepackage{parskip}
  }{% else
    \setlength{\parindent}{0pt}
    \setlength{\parskip}{6pt plus 2pt minus 1pt}}
}{% if KOMA class
  \KOMAoptions{parskip=half}}
\makeatother
\usepackage{xcolor}
\setlength{\emergencystretch}{3em} % prevent overfull lines
\setcounter{secnumdepth}{5}
% Make \paragraph and \subparagraph free-standing
\ifx\paragraph\undefined\else
  \let\oldparagraph\paragraph
  \renewcommand{\paragraph}[1]{\oldparagraph{#1}\mbox{}}
\fi
\ifx\subparagraph\undefined\else
  \let\oldsubparagraph\subparagraph
  \renewcommand{\subparagraph}[1]{\oldsubparagraph{#1}\mbox{}}
\fi


\providecommand{\tightlist}{%
  \setlength{\itemsep}{0pt}\setlength{\parskip}{0pt}}\usepackage{longtable,booktabs,array}
\usepackage{calc} % for calculating minipage widths
% Correct order of tables after \paragraph or \subparagraph
\usepackage{etoolbox}
\makeatletter
\patchcmd\longtable{\par}{\if@noskipsec\mbox{}\fi\par}{}{}
\makeatother
% Allow footnotes in longtable head/foot
\IfFileExists{footnotehyper.sty}{\usepackage{footnotehyper}}{\usepackage{footnote}}
\makesavenoteenv{longtable}
\usepackage{graphicx}
\makeatletter
\def\maxwidth{\ifdim\Gin@nat@width>\linewidth\linewidth\else\Gin@nat@width\fi}
\def\maxheight{\ifdim\Gin@nat@height>\textheight\textheight\else\Gin@nat@height\fi}
\makeatother
% Scale images if necessary, so that they will not overflow the page
% margins by default, and it is still possible to overwrite the defaults
% using explicit options in \includegraphics[width, height, ...]{}
\setkeys{Gin}{width=\maxwidth,height=\maxheight,keepaspectratio}
% Set default figure placement to htbp
\makeatletter
\def\fps@figure{htbp}
\makeatother

\KOMAoption{captions}{tableheading}
\makeatletter
\@ifpackageloaded{tcolorbox}{}{\usepackage[many]{tcolorbox}}
\@ifpackageloaded{fontawesome5}{}{\usepackage{fontawesome5}}
\definecolor{quarto-callout-color}{HTML}{909090}
\definecolor{quarto-callout-note-color}{HTML}{0758E5}
\definecolor{quarto-callout-important-color}{HTML}{CC1914}
\definecolor{quarto-callout-warning-color}{HTML}{EB9113}
\definecolor{quarto-callout-tip-color}{HTML}{00A047}
\definecolor{quarto-callout-caution-color}{HTML}{FC5300}
\definecolor{quarto-callout-color-frame}{HTML}{acacac}
\definecolor{quarto-callout-note-color-frame}{HTML}{4582ec}
\definecolor{quarto-callout-important-color-frame}{HTML}{d9534f}
\definecolor{quarto-callout-warning-color-frame}{HTML}{f0ad4e}
\definecolor{quarto-callout-tip-color-frame}{HTML}{02b875}
\definecolor{quarto-callout-caution-color-frame}{HTML}{fd7e14}
\makeatother
\makeatletter
\makeatother
\makeatletter
\@ifpackageloaded{bookmark}{}{\usepackage{bookmark}}
\makeatother
\makeatletter
\@ifpackageloaded{caption}{}{\usepackage{caption}}
\AtBeginDocument{%
\ifdefined\contentsname
  \renewcommand*\contentsname{Table of contents}
\else
  \newcommand\contentsname{Table of contents}
\fi
\ifdefined\listfigurename
  \renewcommand*\listfigurename{List of Figures}
\else
  \newcommand\listfigurename{List of Figures}
\fi
\ifdefined\listtablename
  \renewcommand*\listtablename{List of Tables}
\else
  \newcommand\listtablename{List of Tables}
\fi
\ifdefined\figurename
  \renewcommand*\figurename{Figure}
\else
  \newcommand\figurename{Figure}
\fi
\ifdefined\tablename
  \renewcommand*\tablename{Table}
\else
  \newcommand\tablename{Table}
\fi
}
\@ifpackageloaded{float}{}{\usepackage{float}}
\floatstyle{ruled}
\@ifundefined{c@chapter}{\newfloat{codelisting}{h}{lop}}{\newfloat{codelisting}{h}{lop}[chapter]}
\floatname{codelisting}{Listing}
\newcommand*\listoflistings{\listof{codelisting}{List of Listings}}
\makeatother
\makeatletter
\@ifpackageloaded{caption}{}{\usepackage{caption}}
\@ifpackageloaded{subcaption}{}{\usepackage{subcaption}}
\makeatother
\makeatletter
\@ifpackageloaded{tcolorbox}{}{\usepackage[many]{tcolorbox}}
\makeatother
\makeatletter
\@ifundefined{shadecolor}{\definecolor{shadecolor}{rgb}{.97, .97, .97}}
\makeatother
\makeatletter
\makeatother
\ifLuaTeX
  \usepackage{selnolig}  % disable illegal ligatures
\fi
\IfFileExists{bookmark.sty}{\usepackage{bookmark}}{\usepackage{hyperref}}
\IfFileExists{xurl.sty}{\usepackage{xurl}}{} % add URL line breaks if available
\urlstyle{same} % disable monospaced font for URLs
\hypersetup{
  pdftitle={Excel Programming and Automation},
  pdfauthor={Jonas Moss},
  colorlinks=true,
  linkcolor={blue},
  filecolor={Maroon},
  citecolor={Blue},
  urlcolor={Blue},
  pdfcreator={LaTeX via pandoc}}

\title{Excel Programming and Automation}
\author{Jonas Moss}
\date{12/12/22}

\begin{document}
\maketitle
\ifdefined\Shaded\renewenvironment{Shaded}{\begin{tcolorbox}[borderline west={3pt}{0pt}{shadecolor}, sharp corners, boxrule=0pt, enhanced, breakable, frame hidden, interior hidden]}{\end{tcolorbox}}\fi

\renewcommand*\contentsname{Table of contents}
{
\hypersetup{linkcolor=}
\setcounter{tocdepth}{2}
\tableofcontents
}
\bookmarksetup{startatroot}

\hypertarget{introduction}{%
\chapter*{Introduction}\label{introduction}}
\addcontentsline{toc}{chapter}{Introduction}

\markboth{Introduction}{Introduction}

Welcome to the course
\href{https://programmeinfo.bi.no/nb/kurs/ELE-3915/2023-var}{ELE 3915
Excel Programming and Automation}. You're about to get your hands dirty!
This course is about being able to do cool things Excel and TypeScript.
It should give you the confidence to assert that you can program in
Excel. And it gives you an introduction to TypeScript, a popular
general-purpose programming language used in Microsoft Office on the
web. The slides can be found on
\href{https://github.com/BI-DS/ELE-3915}{Github}, in the
\href{https://github.com/BI-DS/ELE-3915/slides}{\texttt{slides}}
directory.

If you need to get in contact with me, please send an e-mail to
\texttt{jonas.moss@bi.no}. I do not check It's learning often.

\begin{tcolorbox}[enhanced jigsaw, colbacktitle=quarto-callout-warning-color!10!white, bottomrule=.15mm, toprule=.15mm, colback=white, coltitle=black, bottomtitle=1mm, colframe=quarto-callout-warning-color-frame, opacitybacktitle=0.6, titlerule=0mm, left=2mm, title=\textcolor{quarto-callout-warning-color}{\faExclamationTriangle}\hspace{0.5em}{Warning}, opacityback=0, arc=.35mm, leftrule=.75mm, rightrule=.15mm, toptitle=1mm, breakable]

This webpage is being continuously updated during. Most of the chapters
are not finished.

\end{tcolorbox}

\hypertarget{rough-structure}{%
\section*{Rough structure}\label{rough-structure}}
\addcontentsline{toc}{section}{Rough structure}

\markright{Rough structure}

The course is split into into four parts.

\begin{enumerate}
\def\labelenumi{\arabic{enumi}.}
\item
  \textbf{Lecture 1--3}. Basic Excel usage, including formulas such as
  \texttt{SUM} ,\texttt{COUNTIFS}, \texttt{INDEX}, and \texttt{MATCH}.
  We focus on general mathematical formulas and data manipulation, in
  addition to simple formatting. Roughly speaking, these lectures cover
  the sort of Excel you should expect to see in lecacy Excel sheets.
  Most of the curriculum is covered on the excellent webpage
  \href{https://exceljet.net/}{ExcelJet}.
\item
  \textbf{Lecture 4--6}. These lectures covers modern Excel formulas
  using dynamic arrays,
  \href{https://www.microsoft.com/en-us/research/blog/lambda-the-ultimatae-excel-worksheet-function/}{\texttt{LAMBDA}}
  and \href{https://exceljet.net/functions/let-function}{\texttt{LET}}.
  Many of these features are just a couple of years old (as of 2023). I
  would not expect most employers to know about them. These features can
  do quite a bit of what
  \href{https://en.wikipedia.org/wiki/Visual_Basic_for_Applications}{Visual
  Basic for Applications} (VBA) did in legacy worksheets. Programming
  using \texttt{LAMBDA} and \texttt{LET} is difficult, so make sure you
  take the exercises seriously. Again, most of the curriculum is covered
  on \href{https://exceljet.net/}{ExcelJet}.
\item
  \textbf{Lecture 7--9}. Covers power query, power pivot, and DAX. These
  are modern tools for data visualization, data cleaning, and data
  manipulation. These tools are likely to form the core of your
  automation in Excel. They are widely regarded as the most important
  tools in the Excel analysts toolbox.
\item
  \textbf{Lecture 10--14}. These lectures contain an introduction to
  TypeScript and its application in Excel. Take note that we assume no
  previous exposure to programming in this course.
  \href{https://en.wikipedia.org/wiki/TypeScript}{TypeScript} is a
  \emph{typed} modification of
  \href{https://en.wikipedia.org/wiki/JavaScript}{JavaScript}, the
  dominant language in web development, and has application far beyond
  Excel. The lectures will focus on programming in TypeScript in
  general, and the emphasis is on exercises.
\end{enumerate}

\hypertarget{a-note-on-difficulty}{%
\section*{A note on difficulty}\label{a-note-on-difficulty}}
\addcontentsline{toc}{section}{A note on difficulty}

\markright{A note on difficulty}

Most of the course is about programming. Students often find programming
hard. \textbf{Don't expect to be able solve every exercise in 5
minutes!} Solving programming exercises often take a long time, and you
need to persevere.

Be aware that the difficulty of this course is uneven. For instance,
lectures on \texttt{LAMBDA} and \texttt{LET} are harder than the first
three lectures, and the lectures on power query are likely to be easier.
Do not think this course will be a walk in the park.

To become a decent programmer it's a good idea to

\begin{enumerate}
\def\labelenumi{\arabic{enumi}.}
\tightlist
\item
  Do a lot of exercises.
\item
  Spend at least 20 minutes on each exercise before you give up. You
  need \href{https://www.benkuhn.net/thinkrealhard/}{to think really
  hard}. Don't expect to be able to solve the problem without making an
  effort.
\item
  Do the exercise yourself after you have looked at the solution! Close
  the window and do it from memory. It's also a good idea to revisit the
  same exercise later on, e.g.~the next day, to make sure you're able to
  do it.
\item
  Tinker around, either modifying exercises yourself, or with your own
  ideas. If your tinkering leads to something cool, tell me! Use
  \href{https://www.kaggle.com/}{Kaggle} to download data sets to tinker
  with and \href{https://mockaroo.com/}{Mockaroo} to generate fake but
  plausible-looking data sets.
\end{enumerate}

Do not to spend an inordinate amount of time on an exercise before you
check the solution. If you have spent 1 hour on an exercise and haven't
gotten anywhere, it might be smart save yourself some time and look at
the solution.

Moreover, be aware that programming is often \emph{extremely
frustrating}. It's like talking to someone who just simply refuses to
understand what you're saying, no matter how many times you repeat
yourself. \textbf{It's normal and expected to feel frustrated}!

There are many tips online about learning to program, e.g.,
\href{https://www.codingdojo.com/blog/7-tips-learn-programming-faster}{this
collection of tips}. But it mostly boils down to spending a lot of time
solving problems.

\hypertarget{about-this-site}{%
\section*{About this site}\label{about-this-site}}
\addcontentsline{toc}{section}{About this site}

\markright{About this site}

Curious how this site was made? It is written using
\href{https://quarto.org/docs/books}{Quarto books}.

\bookmarksetup{startatroot}

\hypertarget{excel-basics-i-introduction-to-formulas}{%
\chapter{Excel basics (i): Introduction to
formulas}\label{excel-basics-i-introduction-to-formulas}}

\emph{Last updated:} 03.01.2023

\hypertarget{curriculum}{%
\section{Curriculum}\label{curriculum}}

\begin{itemize}
\item
  Basics of Excel, including cells, active cells, and ranges.
  Worksheets, basic formatting, data types and blank cells.
\item
  Understanding what
  \href{https://exceljet.net/articles/excel-formulas-and-functions}{formulas
  and functions} are, including copying of formulas using relative
  references.
\item
  Reading the signature and short documentation of functions inside
  Excel.
\item
  Using
  \href{https://support.microsoft.com/en-us/office/save-time-with-flash-fill-9159216a-75a0-4c11-82e6-8eca29cb3b89}{flash
  fill}.
\item
  Using the operators \texttt{+},\texttt{-},\texttt{/}, and \texttt{*}.
\item
  A couple of keyboard shortcuts.

  \begin{itemize}
  \item
    Press F2 to edit a cell and see its dependencies.
  \item
    Press escape to exit editing a formula.
  \item
    Press tab to auto-complete a formula.
  \item
    Press SHIFT to add contiguous cells to a selection.
  \item
    Press CTRL+DOWN to go to the last non-blank cell in a contiguous
    column. And CTRL+UP to go to the first column-wise, CTRL+LEFT-ARROW
    to go to the first row-wise, and CTRL+RIGHT-ARROW to go to the last
    row-wise.
  \item
    Hold CTRL to add cells to a selection.
  \item
    Press F4 to taggle absolute references.
  \end{itemize}
\item
  Logical values; the functions \texttt{AND}, \texttt{OR}, \texttt{NOT},
  and the operators \texttt{=}, \texttt{\textless{}},
  \texttt{\textgreater{}}, \texttt{\textless{}=} and
  \texttt{\textgreater{}=}.
\item
  Overview of
  \href{https://exceljet.net/articles/excel-formula-errors}{error
  messages}.
\item
  Knowledge of the basic datatypes of Excel (number; text; logical;
  error; array), including how to use the
  \href{https://exceljet.net/functions/type-function\#:~:text=The\%20Excel\%20TYPE\%20function\%20returns,value\%20in\%20a\%20particular\%20cell}{\texttt{TYPE}}
  function.
\end{itemize}

Basic functions using one range as argument.

\begin{longtable}[]{@{}
  >{\raggedright\arraybackslash}p{(\columnwidth - 2\tabcolsep) * \real{0.6033}}
  >{\raggedright\arraybackslash}p{(\columnwidth - 2\tabcolsep) * \real{0.3967}}@{}}
\toprule()
\begin{minipage}[b]{\linewidth}\raggedright
Name
\end{minipage} & \begin{minipage}[b]{\linewidth}\raggedright
Description
\end{minipage} \\
\midrule()
\endhead
\href{https://exceljet.net/functions/sum-function}{\texttt{SUM}} & Sum
all numbers in a range. \\
\href{https://exceljet.net/functions/product-function}{\texttt{PRODUCT}}
& The product of all numbers in a range. \\
\href{https://exceljet.net/functions/max-function}{\texttt{MAX}} & The
maximum of numbers in a range. \\
\href{https://exceljet.net/functions/min-function}{\texttt{MIN}} & The
minimum of numbers in a range. \\
\href{https://exceljet.net/functions/count-function}{\texttt{COUNT}}/\href{https://exceljet.net/functions/counta-function}{\texttt{COUNTA}}/\href{https://exceljet.net/functions/countblank-function}{\texttt{COUNTBLANK}}
& Count the number of cells in a range that contain numbers
(\texttt{COUNT}), are non-empty (\texttt{COUNTA}), or empty
(\texttt{COUNTBLANK}). \\
\href{https://exceljet.net/functions/rows-function}{\texttt{ROWS}} /
\href{https://exceljet.net/functions/cols-function}{\texttt{COLS}} &
Counts the number of rows / columns in a reference. \\
\href{https://exceljet.net/functions/average-function}{\texttt{AVERAGE}}
& The average of the numbers in a range. Empty cells are ignored. \\
\href{https://exceljet.net/functions/median-function}{\texttt{MEDIAN}} &
The median of the numbers in a range. \\
\href{https://exceljet.net/functions/stdev-function}{\texttt{STDEV}} &
The standard deviation of the numbers in a range. Empty cells are
ignored. \\
\href{https://exceljet.net/functions/countif-function}{\texttt{COUNTIF}}
& Counts cells satisfying a criterion. \\
\bottomrule()
\end{longtable}

\hypertarget{lectures-and-exercises}{%
\section{Lectures and exercises}\label{lectures-and-exercises}}

Remember to look at the solutions only after giving the exercises a
serious attempt. Solve the exercises yourself after looking at the
solution.

The lecture slides are
\href{https://rawcdn.githack.com/BI-DS/ELE-3915/65f16fba2d7ce0fe0fb7fc248d6cf812bb9ba321/slides/lecture01.html}{here}.
The Excel sheets used in the lecture, before being filled in,
\href{https://github.com/BI-DS/ELE-3915/blob/main/slides/lecture01.xlsx}{here}.
The lecture notes after being filled in are
\href{https://github.com/BI-DS/ELE-3915/blob/main/slides/lecture01_solutions.xlsx}{here}.

\href{https://github.com/BI-DS/ELE-3915/blob/main/exercises/exercises01.xlsx}{Here}
are the exercises; the solutions can be found
\href{https://github.com/BI-DS/ELE-3915/blob/main/exercises/exercises01_solutions.xlsx}{here}.

\hypertarget{recommended-resources}{%
\section{Recommended resources}\label{recommended-resources}}

There are many excellent video resources for Excel online. The content
of this lecture is pretty standard stuff, and there are probably 100s of
Youtube videoes covering essentially the same content. For instance, the
formulas covered in this lecture are also covered by Kevin Stratvert
\href{https://www.youtube.com/watch?v=Jl0Qk63z2ZY\&ab_channel=KevinStratvert}{here},
but he goes a little further, covering harder formulas too. Leila
Gharani introduces formulas
\href{https://www.youtube.com/watch?v=y1126PQ5zRU\&ab_channel=LeilaGharani}{here}.

There are many introductions to flash fill too,
e.g.~\href{https://www.youtube.com/watch?v=1KimYFzET1w\&ab_channel=LeilaGharani}{this
one}.

There is a staggering number of shortcuts in Excel, see
e.g.~\href{https://support.microsoft.com/en-us/office/keyboard-shortcuts-in-excel-1798d9d5-842a-42b8-9c99-9b7213f0040f}{here}.
It's easy to get overwhelmed by shortcuts, so make sure you don't try to
learn too many at once though!

\bookmarksetup{startatroot}

\hypertarget{excel-basics-ii}{%
\chapter{Excel basics (ii)}\label{excel-basics-ii}}

\hypertarget{curriculum-1}{%
\section{Curriculum}\label{curriculum-1}}

\hypertarget{absolute-and-relative-references.}{%
\subsection{Absolute and relative
references.}\label{absolute-and-relative-references.}}

\begin{itemize}
\tightlist
\item
  Understand the difference between absolute, mixed, and relative
  references and when to use each.
\item
  Use F4 to cycle between the different kinds of references.
\item
  \texttt{LARGE} and \texttt{SMALL} functions.
\end{itemize}

\hypertarget{conditionals-in-excel.}{%
\subsection{Conditionals in Excel.}\label{conditionals-in-excel.}}

\begin{itemize}
\tightlist
\item
  Understand conditional formulas and their uses, such as the
  \texttt{IF} function, nested \texttt{IF}s, and the \texttt{IFS}
  function.
\item
  Learn the conditional aggregation functions such as
  \texttt{SUMIF}/\texttt{SUMIFS}, \texttt{COUNTIF}/\texttt{COUNTIFS},
  and \texttt{MINIFS}/\texttt{MAXIFS} functions.
\end{itemize}

\hypertarget{additional-mathematical-functions}{%
\subsection{Additional mathematical
functions}\label{additional-mathematical-functions}}

\begin{itemize}
\tightlist
\item
  Most of the math functions are relevant to us. We won't cover all of
  them in detail. Here is a list of particularly important functions.
\end{itemize}

\begin{longtable}[]{@{}
  >{\raggedright\arraybackslash}p{(\columnwidth - 2\tabcolsep) * \real{0.3056}}
  >{\raggedright\arraybackslash}p{(\columnwidth - 2\tabcolsep) * \real{0.6944}}@{}}
\toprule()
\begin{minipage}[b]{\linewidth}\raggedright
Name
\end{minipage} & \begin{minipage}[b]{\linewidth}\raggedright
Description
\end{minipage} \\
\midrule()
\endhead
\texttt{ABS} & Find the absolute value of a number. \\
\texttt{EXP} & Find the value of e raised to the power of a number. \\
\texttt{LOG} & Get the logarithm of a number. \\
\texttt{FLOOR.MATH} & Round number down to nearest multiple. \\
\texttt{CEILING.MATH} & Round a number up to nearest multiple. \\
\texttt{ROUND} & Round a number to a given number of digits. \\
\texttt{COMBIN} & Get number of combinations without repetitions. \\
\texttt{MOD} / \texttt{QUOTIENT} & Get the remainder from division /
Returns the quotient without a remainder. \\
\texttt{POWER} / \texttt{\^{}} operator / \texttt{SQRT} & Raise a number
to a power or calculate the square root. \\
\bottomrule()
\end{longtable}

\begin{itemize}
\tightlist
\item
  Learn about how and when to use the \texttt{SUMPRODUCT} function.
\end{itemize}

\hypertarget{working-with-text}{%
\subsection{Working with text}\label{working-with-text}}

\begin{itemize}
\tightlist
\item
  Understand the use of the \texttt{\&} operator for concatenation.
\item
  Learn to solve basic text manipulation tasks.
\item
  Get an overview of the text manipulation functions in Excel and how to
  apply them.
\end{itemize}

\begin{longtable}[]{@{}
  >{\raggedright\arraybackslash}p{(\columnwidth - 2\tabcolsep) * \real{0.3472}}
  >{\raggedright\arraybackslash}p{(\columnwidth - 2\tabcolsep) * \real{0.6528}}@{}}
\toprule()
\begin{minipage}[b]{\linewidth}\raggedright
Name
\end{minipage} & \begin{minipage}[b]{\linewidth}\raggedright
Description
\end{minipage} \\
\midrule()
\endhead
\texttt{TRIM} & Remove extra spaces from text. \\
\texttt{TEXTBEFORE} / \texttt{TEXTAFTER} & Extract text before / after
delimiter. \\
\texttt{EXACT} & Compare two text strings, taking case into account. \\
\texttt{SUBSTITUTE} & Replace text based on content. \\
\texttt{LOWER}/\texttt{UPPER} & Transform text to lower/upper case. \\
\texttt{PROPER} & Capitalize first letter of each word in text. \\
\texttt{CONCAT} / \texttt{TEXTJOIN} & Join text values with(out) a
delimiter. \\
\texttt{LEN} & Get the length of the text \\
\texttt{LEFT} / \texttt{MID} / \texttt{RIGHT} & Extract text from the
left/middle/right of a string \\
\texttt{FIND} & Get location substring in a string. \\
\texttt{REPLACE} & Replace text based on location. \\
\bottomrule()
\end{longtable}

\begin{itemize}
\tightlist
\item
  Note that not all useful text manipulation tools are built-in and some
  tasks, such as counting the number of words in a text, may require
  creating custom functions.
\item
  Some of the functions are outdated. You will see them used, but modern
  alternatives are better. This is arguably the case for \texttt{LEFT};
  \texttt{MID}, and \texttt{RIGHT}, \texttt{FIND} and \texttt{SEARCH}.
  The modern variants will be covered in later lectures.
\end{itemize}

\hypertarget{lectures-and-exercises-1}{%
\section{Lectures and exercises}\label{lectures-and-exercises-1}}

The Excel sheets used in the lecture, before being filled in,
\href{https://github.com/BI-DS/ELE-3915/blob/main/slides/lecture02.xlsx}{here}.
The lecture notes after being filled in are
\href{https://github.com/BI-DS/ELE-3915/blob/main/slides/lecture02_solutions.xlsx}{here}.
\href{https://github.com/BI-DS/ELE-3915/blob/main/exercises/exercises02.xlsx}{Here}
are the exercises; the solutions can be found
\href{https://github.com/BI-DS/ELE-3915/blob/main/exercises/exercises02_solutions.xlsx}{here}.

\hypertarget{recommended-resources-1}{%
\section{Recommended resources}\label{recommended-resources-1}}

The YouTuber
\href{https://www.youtube.com/watch?v=thvE8Eg-Pqg\&ab_channel=Chandoo}{Chandoo}
discusses the text manipulation functions at length, so does the
YouTuber Leila Gharani
\href{https://www.youtube.com/@LeilaGharani}{Leila Gharani}. There are
many good videos about these topics, and I strongly encourage you to
explore them.

\bookmarksetup{startatroot}

\hypertarget{excel-basics-iii-lookups-and-aggregation}{%
\chapter{Excel basics (iii): Lookups and
aggregation}\label{excel-basics-iii-lookups-and-aggregation}}

\hypertarget{curriculum-2}{%
\section{Curriculum}\label{curriculum-2}}

\hypertarget{lookup-aggregation-and-filtering.}{%
\subsection{Lookup, aggregation, and
filtering.}\label{lookup-aggregation-and-filtering.}}

\begin{itemize}
\tightlist
\item
  Knowledge of the \texttt{INDEX} functions, \texttt{XMATCH} function,
  and how they work together.
\item
  Understanding the modern look up function \texttt{XLOOKUP}.
\item
  Some uses of the important functions \texttt{UNIQUE} and
  \texttt{FILTER}.
\item
  Using \texttt{FILTER} in data aggregation tasks.
\item
  Gain an understanding of how to use the lookup functions,
  \texttt{FILTER}, and \texttt{UNIQUE} to perform tasks of intermediate
  complexity.
\end{itemize}

\hypertarget{miscellaneous}{%
\subsection{Miscellaneous}\label{miscellaneous}}

\begin{itemize}
\tightlist
\item
  Introduction to \href{https://exceljet.net/articles/excel-tables}{data
  tables}. This article contains more information than I went through in
  the lecture, and you should read it.
\item
  We have a cursory look at conditional formatting and very basic
  charts.
\item
  In addition, we consider basic use of data validation, i.e., making
  drop down lists.
\item
  Ranking and ordering using \texttt{RANK.EQ}, \texttt{RANK.AVG},
  \texttt{SORT}, and \texttt{SORTBY}. Knowledge of sorting using Excel
  buttons.
\end{itemize}

\hypertarget{lectures-and-exercises-2}{%
\section{Lectures and exercises}\label{lectures-and-exercises-2}}

The Excel sheets used in the lecture, before being filled in,
\href{https://github.com/BI-DS/ELE-3915/blob/main/slides/lecture03.xlsx}{here}.
The lecture notes after being filled in are
\href{https://github.com/BI-DS/ELE-3915/blob/main/slides/lecture03_solutions.xlsx}{here}.
\href{https://github.com/BI-DS/ELE-3915/blob/main/exercises/exercises03.xlsx}{Here}
are the exercises; the solutions can be found
\href{https://github.com/BI-DS/ELE-3915/blob/main/exercises/exercises03_solutions.xlsx}{here}.

\hypertarget{recommended-resources-2}{%
\section{Recommended resources}\label{recommended-resources-2}}

The
\href{https://www.wiseowl.co.uk/excel/exercises/standard/lookup-functions/}{Wise
Owl Training} site has several exercises on lookup functions. They also
have exercises covering other Excel topics.

You will be exposed to the function \texttt{VLOOKUP} outside of this
course. It's worth it to spend some minutes understanding the difference
between \texttt{XLOOKUP} and \texttt{VLOOKUP}, see
e.g.~\href{https://www.youtube.com/watch?v=HXU7lsd0ftA}{this video}.

Data validation and conditional formatting are bigger topics that it
would seem from our coverage. You can read more about them on ExcelJet
or elsewhere.

\bookmarksetup{startatroot}

\hypertarget{programming-in-excel-i-introduction-to-lambda}{%
\chapter{\texorpdfstring{Programming in Excel (i): Introduction to
\texttt{LAMBDA}}{Programming in Excel (i): Introduction to LAMBDA}}\label{programming-in-excel-i-introduction-to-lambda}}

\hypertarget{curriculum-3}{%
\section{Curriculum}\label{curriculum-3}}

\begin{itemize}
\item
  How to define your own functions using \texttt{LAMBDA} and why it is
  useful.
\item
  Using the name manager.
\item
  Introduction to the
  \href{https://www.microsoft.com/en-us/garage/profiles/advanced-formula-environment-a-microsoft-garage-project/}{Advanced
  Formula Environment}, with emphasis on how to write documented and
  reusable code using the module functionality.
\item
  Construction of custom arrays with \texttt{\{\}}.
\item
  What a
  \href{https://exceljet.net/articles/dynamic-array-formulas-in-excel}{dynamic
  arrays}, what the \texttt{\#SPILL!} error means, and how to use
  dynamic arrays are ranges.
\item
  Extending the basic Excel functions with \texttt{MAP}, \texttt{BYROW},
  \texttt{BYCOLUMN}.
\item
  Exercise in making custom functions using the tools we have learned
  until now.
\item
  Making random numbers using the function \texttt{RANDARRAY}.
\end{itemize}

\hypertarget{lectures-and-exercises-3}{%
\section{Lectures and exercises}\label{lectures-and-exercises-3}}

The Excel sheets used in the lecture, before being filled in,
\href{https://github.com/BI-DS/ELE-3915/blob/main/slides/lecture04.xlsx}{here}.
The lecture notes after being filled in are
\href{https://github.com/BI-DS/ELE-3915/blob/main/slides/lecture04_solutions.xlsx}{here}.
\href{https://github.com/BI-DS/ELE-3915/blob/main/exercises/exercises04.xlsx}{Here}
are the exercises; the solutions can be found
\href{https://github.com/BI-DS/ELE-3915/blob/main/exercises/exercises04_solutions.xlsx}{here}.

\hypertarget{recommended-resources-3}{%
\section{Recommended resources}\label{recommended-resources-3}}

\bookmarksetup{startatroot}

\hypertarget{programming-in-excel-ii-array-functions-let-and-indirect.}{%
\chapter{\texorpdfstring{Programming in Excel (ii): Array functions,
\texttt{LET}, and
\texttt{INDIRECT}.}{Programming in Excel (ii): Array functions, LET, and INDIRECT.}}\label{programming-in-excel-ii-array-functions-let-and-indirect.}}

\begin{tcolorbox}[enhanced jigsaw, colbacktitle=quarto-callout-warning-color!10!white, bottomrule=.15mm, toprule=.15mm, colback=white, coltitle=black, bottomtitle=1mm, colframe=quarto-callout-warning-color-frame, opacitybacktitle=0.6, titlerule=0mm, left=2mm, title=\textcolor{quarto-callout-warning-color}{\faExclamationTriangle}\hspace{0.5em}{Warning}, opacityback=0, arc=.35mm, leftrule=.75mm, rightrule=.15mm, toptitle=1mm, breakable]

(7/2 2023) The lecture notes, exercises, and curricula are finished, but
not all the videos.

\end{tcolorbox}

\hypertarget{curriculum-4}{%
\section{Curriculum}\label{curriculum-4}}

Some of the lectures below contain exercises. These are designed to be
done immediately after watching the video.

\begin{enumerate}
\def\labelenumi{\arabic{enumi}.}
\item
  Make \href{https://exceljet.net/glossary/array-constant}{array
  constants} using \texttt{\{...\}} Array constants are custom made
  arrays. They can be rows, columns, or two-dimensional. \emph{Links:}
  \href{https://github.com/BI-DS/ELE-3915/blob/main/slides/lecture05/array_constants.xlsx}{Lecture
  notes} and video.
\item
  Select rows or columns using numerical indexing with
  \href{https://exceljet.net/functions/chooserows-function}{\texttt{CHOOSEROWS}}
  and
  \href{https://exceljet.net/functions/choosecols-function}{\texttt{CHOOSECOLS}}.
  \emph{Links:}
  \href{https://github.com/BI-DS/ELE-3915/blob/main/slides/lecture05/choosecols_chooserows.xlsx}{Lecture
  notes} and video.
\item
  Keep or remove rows/columns with
  \href{https://exceljet.net/functions/take-function}{\texttt{TAKE}} and
  \href{https://exceljet.net/functions/drop-function}{\texttt{DROP}}.
  \emph{Links:}
  \href{https://github.com/BI-DS/ELE-3915/blob/main/slides/lecture05/take_drop.xlsx}{Lecture
  notes} and video.
\item
  Manipulate array dimensions with
  \href{https://exceljet.net/functions/torow-function}{\texttt{TOROW}}
  and
  \href{https://exceljet.net/functions/tocol-function}{\texttt{TOCOL}}.
  \texttt{TOROW} transforms an array into a row and \texttt{TOCOLS}
  transforms an array into a column. These are useful for presenting
  data and important when creating functions, as some functions such as
  \texttt{FILTER} do not always work as intended when applied to
  two-dimensional arrays. \emph{Links:}
  \href{https://github.com/BI-DS/ELE-3915/blob/main/slides/lecture05/torow_tocol.xlsx}{Lecture
  notes} and video.
\item
  Construct arrays with
  \href{https://exceljet.net/functions/hstack-function}{\texttt{HSTACK}}
  and
  \href{https://exceljet.net/functions/vstack-function}{\texttt{VSTACK}}.
  \texttt{HSTACK} stacks arrays horizontally and \texttt{VSTACK} stacks
  arrays vertically. These functions are especially handy when making
  functions. \emph{Links:}
  \href{https://github.com/BI-DS/ELE-3915/blob/main/slides/lecture05/hstack_vstack.xlsx}{Lecture
  notes} and video.
\item
  Make arrays more presentable with
  \href{https://exceljet.net/functions/wrapcols-function}{\texttt{WRAPCOLS}}
  and
  \href{https://exceljet.net/functions/wraprows-function}{\texttt{WRAPROWS}}.
  \emph{Links:}
  \href{https://github.com/BI-DS/ELE-3915/blob/main/slides/lecture05/wrapcols_wraprows.xlsx}{Lecture
  notes} and video.
\item
  Make formulas shorter with
  \href{https://exceljet.net/functions/let-function}{\texttt{LET}}. The
  \texttt{LET} lets us define \emph{local variables}, helping us reduce
  repetition in formulas. \emph{Links:}
  \href{https://github.com/BI-DS/ELE-3915/blob/main/slides/lecture05/let_simple_formulas.xlsx}{Lecture
  notes} and video.
\item
  Make complex functions with \texttt{LET}. The \texttt{LET} function is
  especially useful when making functions. We'll have a look at several
  examples. \emph{Links:}
  \href{https://github.com/BI-DS/ELE-3915/blob/main/slides/lecture05/let_complex_functions.xlsx}{Lecture
  notes} and video.
\item
  Use the
  \href{https://exceljet.net/functions/indirect-function}{\texttt{INDIRECT}}
  function to turn text into references. This function is function is
  widely used, but has largely been supplanted by modern array
  functions. \emph{Links:}
  \href{https://github.com/BI-DS/ELE-3915/blob/main/slides/lecture05/indirect_function.xlsx}{Lecture
  notes} and video.
\item
  \textbf{Application:} We define our own variant of the net present
  value function, called \texttt{NPV} in Excel. \emph{Links:}
  \href{https://github.com/BI-DS/ELE-3915/blob/main/slides/lecture05/your_own_npv.xlsx}{Lecture
  notes} and video.
\item
  \textbf{Application:} Summarizing tables with drop-down lists.
  \emph{Links:}
  \href{https://github.com/BI-DS/ELE-3915/blob/main/slides/lecture05/summarizing_tables.xlsx}{Lecture
  notes} and video.
\end{enumerate}

\hypertarget{lectures-and-exercises-4}{%
\section{Lectures and exercises}\label{lectures-and-exercises-4}}

The lecture notes and videos for each curriculum point can be found
above. The main exercises are
\href{https://github.com/BI-DS/ELE-3915/blob/main/exercises/exercises05.xlsx}{here},
with solution proposals included.

\bookmarksetup{startatroot}

\hypertarget{programming-in-excel-iii-reduce-scan-and-applications}{%
\chapter{\texorpdfstring{Programming in Excel (iii): \texttt{REDUCE},
\texttt{SCAN}, and
applications}{Programming in Excel (iii): REDUCE, SCAN, and applications}}\label{programming-in-excel-iii-reduce-scan-and-applications}}

\hypertarget{curriculum-5}{%
\section{Curriculum}\label{curriculum-5}}

Excel is equipped with many functions for handling dynamic arrays and
\texttt{LAMBDA} functions. The most important are \texttt{FILTER},
\texttt{MAP} (and its cousins \texttt{BYCOL}, \texttt{BYROW}, and
\texttt{MAKEARRAY}), and the pair \texttt{REDUCE} and \texttt{SCAN}.

The \texttt{REDUCE} and \texttt{SCAN} functions are somewhat harder to
understand than most of the functions covered until now, and we will
spend some time explaining them.

We cover four functions in this lecture.

\begin{itemize}
\tightlist
\item
  \texttt{OFFSET}, an function used to reference ranges. In most cases
  it's possible to use INDEX instead.
  \href{https://github.com/BI-DS/ELE-3915/blob/main/slides/lecture06/01_offset.xlsx}{Lecture
  notes.}
\item
  \texttt{ISOMITTED} is used for optional arguments in \texttt{LAMBDA}
  functions.
  \href{https://github.com/BI-DS/ELE-3915/blob/main/slides/lecture06/02_optional_arguments.xlsx}{Lecture
  notes.}
\item
  \texttt{REDUCE}
  \href{https://github.com/BI-DS/ELE-3915/blob/main/slides/lecture06/03_reduce.xlsx}{Lecture
  notes.}

  \begin{enumerate}
  \def\labelenumi{\arabic{enumi}.}
  \item
    \href{https://www.youtube.com/watch?v=Gt7_BtFYfBw\&ab_channel=LearnGoogleSheets\%26ExcelSpreadsheets}{Excel
    REDUCE Function - LAMBDA Array Formulas in Excel \& Google Sheets}
  \item
    Reduces an array into a single output.
  \item
    Generalizes \texttt{SUM} and \texttt{PRODUCT}.
  \end{enumerate}
\item
  \texttt{SCAN}.
  \href{https://github.com/BI-DS/ELE-3915/blob/main/slides/lecture06/05_scan.xlsx}{Lecture
  notes.}

  \begin{enumerate}
  \def\labelenumi{\arabic{enumi}.}
  \item
    Reduces an array, but keeps the output for all consecutive
    subarrays.
  \item
    Generalizes the cumulative \texttt{SUM}.
  \end{enumerate}
\end{itemize}

We will also discuss four applications of the methods covered so far:

\begin{enumerate}
\def\labelenumi{\arabic{enumi}.}
\tightlist
\item
  The FILTER-MAP-REDUCE pattern. Most data analysis tasks can be done
  using a combination of \texttt{FILTER}, \texttt{MAP}, and
  \texttt{REDUCE/SCAN}, where array manipulation
  (e.g.~\texttt{CHOOSEROWS} and \texttt{HSTACK}) and local variable
  (\texttt{LET}) are used as needed.
  \href{https://github.com/BI-DS/ELE-3915/blob/main/slides/lecture06/04_map-filter-reduce.xlsx}{Lecture
  notes.}
\item
  \texttt{TAKEWHILE}: A function that takes rows from an array until a
  predicate vector is \texttt{FALSE}.
  \href{https://github.com/BI-DS/ELE-3915/blob/main/slides/lecture06/06_application_1_takewhile.xlsx}{Lecture
  notes.}
\item
  \texttt{GROUPBY}: A pattern used to group and aggregate data.
  \href{https://github.com/BI-DS/ELE-3915/blob/main/slides/lecture06/07_application_2_groupby.xlsx}{Lecture
  notes.}
\item
  Dynamic dropdows: How to make multiple dropdown lists where one
  depends on the other.
  \href{https://github.com/BI-DS/ELE-3915/blob/main/slides/lecture06/08_application_3_dynamic_dropdowns.xlsx}{Lecture
  notes.}
\end{enumerate}

\hypertarget{exercises}{%
\section{Exercises}\label{exercises}}

Most of the lecture notes above contain small exercises. I recommend you
do them before attempting starting on the main exercise sheet. The main
exercises are
\href{https://github.com/BI-DS/ELE-3915/blob/main/exercises/exercises06.xlsx}{here},
with solution proposals included. \textbf{This exercise sheet is NOT
completely finished yet; I haven't written solutions to all the
exercises, and sometimes the solution is inside the main sheet! (14/2)}.

\bookmarksetup{startatroot}

\hypertarget{power-pivot-and-dax}{%
\chapter{Power pivot and DAX}\label{power-pivot-and-dax}}

\begin{tcolorbox}[enhanced jigsaw, colbacktitle=quarto-callout-important-color!10!white, bottomrule=.15mm, toprule=.15mm, colback=white, coltitle=black, bottomtitle=1mm, colframe=quarto-callout-important-color-frame, opacitybacktitle=0.6, titlerule=0mm, left=2mm, title=\textcolor{quarto-callout-important-color}{\faExclamation}\hspace{0.5em}{Important}, opacityback=0, arc=.35mm, leftrule=.75mm, rightrule=.15mm, toptitle=1mm, breakable]

This page is \textbf{\emph{not finished}}, but might contain notes from
the course developers. The curriculum, exercises, and recommended
resources listed on this page is subject to change.

\end{tcolorbox}

\hypertarget{curriculum-6}{%
\section{Curriculum}\label{curriculum-6}}

\hypertarget{exercises-1}{%
\section{Exercises}\label{exercises-1}}

\hypertarget{recommended-resources-4}{%
\section{Recommended resources}\label{recommended-resources-4}}

\bookmarksetup{startatroot}

\hypertarget{typescript-i}{%
\chapter{Typescript (i)}\label{typescript-i}}

\begin{tcolorbox}[enhanced jigsaw, colbacktitle=quarto-callout-important-color!10!white, bottomrule=.15mm, toprule=.15mm, colback=white, coltitle=black, bottomtitle=1mm, colframe=quarto-callout-important-color-frame, opacitybacktitle=0.6, titlerule=0mm, left=2mm, title=\textcolor{quarto-callout-important-color}{\faExclamation}\hspace{0.5em}{Important}, opacityback=0, arc=.35mm, leftrule=.75mm, rightrule=.15mm, toptitle=1mm, breakable]

This page is \textbf{\emph{not finished}}, but might contain notes from
the course developers. The curriculum, exercises, and recommended
resources listed on this page is subject to change.

\end{tcolorbox}

\hypertarget{curriculum-7}{%
\section{Curriculum}\label{curriculum-7}}

\begin{itemize}
\tightlist
\item
  \textbf{Primitive JavaScript values.} A list can be found
  \href{https://developer.mozilla.org/en-US/docs/Web/JavaScript/Data_structures}{here}.

  \begin{itemize}
  \tightlist
  \item
    \texttt{Undefined} and \texttt{Null}.
  \item
    Core primitives: \texttt{Boolean}, \texttt{Number}, and
    \texttt{String}.
  \item
    Specialized: \texttt{BigInt} and \texttt{Symbol}.
  \end{itemize}
\item
  \textbf{Arrays:} Unnamed collection of objects.
\item
  \textbf{Functions:} Takes input vales and does something with them.
\end{itemize}

\hypertarget{exercises-2}{%
\section{Exercises}\label{exercises-2}}

\hypertarget{recommended-resources-5}{%
\section{Recommended resources}\label{recommended-resources-5}}

\bookmarksetup{startatroot}

\hypertarget{typescript-ii}{%
\chapter{Typescript (ii)}\label{typescript-ii}}

\begin{tcolorbox}[enhanced jigsaw, colbacktitle=quarto-callout-important-color!10!white, bottomrule=.15mm, toprule=.15mm, colback=white, coltitle=black, bottomtitle=1mm, colframe=quarto-callout-important-color-frame, opacitybacktitle=0.6, titlerule=0mm, left=2mm, title=\textcolor{quarto-callout-important-color}{\faExclamation}\hspace{0.5em}{Important}, opacityback=0, arc=.35mm, leftrule=.75mm, rightrule=.15mm, toptitle=1mm, breakable]

This page is \textbf{\emph{not finished}}, but might contain notes from
the course developers. The curriculum, exercises, and recommended
resources listed on this page is subject to change.

\end{tcolorbox}

\hypertarget{curriculum-8}{%
\section{Curriculum}\label{curriculum-8}}

\hypertarget{exercises-3}{%
\section{Exercises}\label{exercises-3}}

\hypertarget{recommended-resources-6}{%
\section{Recommended resources}\label{recommended-resources-6}}

\bookmarksetup{startatroot}

\hypertarget{typescript-iii}{%
\chapter{Typescript (iii)}\label{typescript-iii}}

\begin{tcolorbox}[enhanced jigsaw, colbacktitle=quarto-callout-important-color!10!white, bottomrule=.15mm, toprule=.15mm, colback=white, coltitle=black, bottomtitle=1mm, colframe=quarto-callout-important-color-frame, opacitybacktitle=0.6, titlerule=0mm, left=2mm, title=\textcolor{quarto-callout-important-color}{\faExclamation}\hspace{0.5em}{Important}, opacityback=0, arc=.35mm, leftrule=.75mm, rightrule=.15mm, toptitle=1mm, breakable]

This page is \textbf{\emph{not finished}}, but might contain notes from
the course developers. The curriculum, exercises, and recommended
resources listed on this page is subject to change.

\end{tcolorbox}

\hypertarget{curriculum-9}{%
\section{Curriculum}\label{curriculum-9}}

\hypertarget{exercises-4}{%
\section{Exercises}\label{exercises-4}}

\hypertarget{recommended-resources-7}{%
\section{Recommended resources}\label{recommended-resources-7}}

\bookmarksetup{startatroot}

\hypertarget{typescript-iv}{%
\chapter{Typescript (iv)}\label{typescript-iv}}

\begin{tcolorbox}[enhanced jigsaw, colbacktitle=quarto-callout-important-color!10!white, bottomrule=.15mm, toprule=.15mm, colback=white, coltitle=black, bottomtitle=1mm, colframe=quarto-callout-important-color-frame, opacitybacktitle=0.6, titlerule=0mm, left=2mm, title=\textcolor{quarto-callout-important-color}{\faExclamation}\hspace{0.5em}{Important}, opacityback=0, arc=.35mm, leftrule=.75mm, rightrule=.15mm, toptitle=1mm, breakable]

This page is \textbf{\emph{not finished}}, but might contain notes from
the course developers. The curriculum, exercises, and recommended
resources listed on this page is subject to change.

\end{tcolorbox}

\hypertarget{curriculum-10}{%
\section{Curriculum}\label{curriculum-10}}

\hypertarget{exercises-5}{%
\section{Exercises}\label{exercises-5}}

\hypertarget{recommended-resources-8}{%
\section{Recommended resources}\label{recommended-resources-8}}

\bookmarksetup{startatroot}

\hypertarget{typescript-in-excel}{%
\chapter{TypeScript in Excel}\label{typescript-in-excel}}

\begin{tcolorbox}[enhanced jigsaw, colbacktitle=quarto-callout-important-color!10!white, bottomrule=.15mm, toprule=.15mm, colback=white, coltitle=black, bottomtitle=1mm, colframe=quarto-callout-important-color-frame, opacitybacktitle=0.6, titlerule=0mm, left=2mm, title=\textcolor{quarto-callout-important-color}{\faExclamation}\hspace{0.5em}{Important}, opacityback=0, arc=.35mm, leftrule=.75mm, rightrule=.15mm, toptitle=1mm, breakable]

This page is \textbf{\emph{not finished}}, but might contain notes from
the course developers. The curriculum, exercises, and recommended
resources listed on this page is subject to change.

\end{tcolorbox}

\hypertarget{curriculum-11}{%
\section{Curriculum}\label{curriculum-11}}

\hypertarget{exercises-6}{%
\section{Exercises}\label{exercises-6}}

\hypertarget{solutions-to-exercises}{%
\section{Solutions to exercises}\label{solutions-to-exercises}}

\hypertarget{recommended-resources-9}{%
\section{Recommended resources}\label{recommended-resources-9}}



\end{document}
